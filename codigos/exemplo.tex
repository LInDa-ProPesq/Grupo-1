\documentclass[]{article}
\usepackage{lmodern}
\usepackage{amssymb,amsmath}
\usepackage{ifxetex,ifluatex}
\usepackage{fixltx2e} % provides \textsubscript
\ifnum 0\ifxetex 1\fi\ifluatex 1\fi=0 % if pdftex
  \usepackage[T1]{fontenc}
  \usepackage[utf8]{inputenc}
\else % if luatex or xelatex
  \ifxetex
    \usepackage{mathspec}
  \else
    \usepackage{fontspec}
  \fi
  \defaultfontfeatures{Ligatures=TeX,Scale=MatchLowercase}
\fi
% use upquote if available, for straight quotes in verbatim environments
\IfFileExists{upquote.sty}{\usepackage{upquote}}{}
% use microtype if available
\IfFileExists{microtype.sty}{%
\usepackage{microtype}
\UseMicrotypeSet[protrusion]{basicmath} % disable protrusion for tt fonts
}{}
\usepackage[margin=1in]{geometry}
\usepackage{hyperref}
\hypersetup{unicode=true,
            pdftitle={markdown\_matheus},
            pdfborder={0 0 0},
            breaklinks=true}
\urlstyle{same}  % don't use monospace font for urls
\usepackage{color}
\usepackage{fancyvrb}
\newcommand{\VerbBar}{|}
\newcommand{\VERB}{\Verb[commandchars=\\\{\}]}
\DefineVerbatimEnvironment{Highlighting}{Verbatim}{commandchars=\\\{\}}
% Add ',fontsize=\small' for more characters per line
\usepackage{framed}
\definecolor{shadecolor}{RGB}{248,248,248}
\newenvironment{Shaded}{\begin{snugshade}}{\end{snugshade}}
\newcommand{\KeywordTok}[1]{\textcolor[rgb]{0.13,0.29,0.53}{\textbf{#1}}}
\newcommand{\DataTypeTok}[1]{\textcolor[rgb]{0.13,0.29,0.53}{#1}}
\newcommand{\DecValTok}[1]{\textcolor[rgb]{0.00,0.00,0.81}{#1}}
\newcommand{\BaseNTok}[1]{\textcolor[rgb]{0.00,0.00,0.81}{#1}}
\newcommand{\FloatTok}[1]{\textcolor[rgb]{0.00,0.00,0.81}{#1}}
\newcommand{\ConstantTok}[1]{\textcolor[rgb]{0.00,0.00,0.00}{#1}}
\newcommand{\CharTok}[1]{\textcolor[rgb]{0.31,0.60,0.02}{#1}}
\newcommand{\SpecialCharTok}[1]{\textcolor[rgb]{0.00,0.00,0.00}{#1}}
\newcommand{\StringTok}[1]{\textcolor[rgb]{0.31,0.60,0.02}{#1}}
\newcommand{\VerbatimStringTok}[1]{\textcolor[rgb]{0.31,0.60,0.02}{#1}}
\newcommand{\SpecialStringTok}[1]{\textcolor[rgb]{0.31,0.60,0.02}{#1}}
\newcommand{\ImportTok}[1]{#1}
\newcommand{\CommentTok}[1]{\textcolor[rgb]{0.56,0.35,0.01}{\textit{#1}}}
\newcommand{\DocumentationTok}[1]{\textcolor[rgb]{0.56,0.35,0.01}{\textbf{\textit{#1}}}}
\newcommand{\AnnotationTok}[1]{\textcolor[rgb]{0.56,0.35,0.01}{\textbf{\textit{#1}}}}
\newcommand{\CommentVarTok}[1]{\textcolor[rgb]{0.56,0.35,0.01}{\textbf{\textit{#1}}}}
\newcommand{\OtherTok}[1]{\textcolor[rgb]{0.56,0.35,0.01}{#1}}
\newcommand{\FunctionTok}[1]{\textcolor[rgb]{0.00,0.00,0.00}{#1}}
\newcommand{\VariableTok}[1]{\textcolor[rgb]{0.00,0.00,0.00}{#1}}
\newcommand{\ControlFlowTok}[1]{\textcolor[rgb]{0.13,0.29,0.53}{\textbf{#1}}}
\newcommand{\OperatorTok}[1]{\textcolor[rgb]{0.81,0.36,0.00}{\textbf{#1}}}
\newcommand{\BuiltInTok}[1]{#1}
\newcommand{\ExtensionTok}[1]{#1}
\newcommand{\PreprocessorTok}[1]{\textcolor[rgb]{0.56,0.35,0.01}{\textit{#1}}}
\newcommand{\AttributeTok}[1]{\textcolor[rgb]{0.77,0.63,0.00}{#1}}
\newcommand{\RegionMarkerTok}[1]{#1}
\newcommand{\InformationTok}[1]{\textcolor[rgb]{0.56,0.35,0.01}{\textbf{\textit{#1}}}}
\newcommand{\WarningTok}[1]{\textcolor[rgb]{0.56,0.35,0.01}{\textbf{\textit{#1}}}}
\newcommand{\AlertTok}[1]{\textcolor[rgb]{0.94,0.16,0.16}{#1}}
\newcommand{\ErrorTok}[1]{\textcolor[rgb]{0.64,0.00,0.00}{\textbf{#1}}}
\newcommand{\NormalTok}[1]{#1}
\usepackage{graphicx,grffile}
\makeatletter
\def\maxwidth{\ifdim\Gin@nat@width>\linewidth\linewidth\else\Gin@nat@width\fi}
\def\maxheight{\ifdim\Gin@nat@height>\textheight\textheight\else\Gin@nat@height\fi}
\makeatother
% Scale images if necessary, so that they will not overflow the page
% margins by default, and it is still possible to overwrite the defaults
% using explicit options in \includegraphics[width, height, ...]{}
\setkeys{Gin}{width=\maxwidth,height=\maxheight,keepaspectratio}
\IfFileExists{parskip.sty}{%
\usepackage{parskip}
}{% else
\setlength{\parindent}{0pt}
\setlength{\parskip}{6pt plus 2pt minus 1pt}
}
\setlength{\emergencystretch}{3em}  % prevent overfull lines
\providecommand{\tightlist}{%
  \setlength{\itemsep}{0pt}\setlength{\parskip}{0pt}}
\setcounter{secnumdepth}{0}
% Redefines (sub)paragraphs to behave more like sections
\ifx\paragraph\undefined\else
\let\oldparagraph\paragraph
\renewcommand{\paragraph}[1]{\oldparagraph{#1}\mbox{}}
\fi
\ifx\subparagraph\undefined\else
\let\oldsubparagraph\subparagraph
\renewcommand{\subparagraph}[1]{\oldsubparagraph{#1}\mbox{}}
\fi

%%% Use protect on footnotes to avoid problems with footnotes in titles
\let\rmarkdownfootnote\footnote%
\def\footnote{\protect\rmarkdownfootnote}

%%% Change title format to be more compact
\usepackage{titling}

% Create subtitle command for use in maketitle
\newcommand{\subtitle}[1]{
  \posttitle{
    \begin{center}\large#1\end{center}
    }
}

\setlength{\droptitle}{-2em}

  \title{markdown\_matheus}
    \pretitle{\vspace{\droptitle}\centering\huge}
  \posttitle{\par}
    \author{}
    \preauthor{}\postauthor{}
    \date{}
    \predate{}\postdate{}
  

\begin{document}
\maketitle

\subsection{R Markdown}\label{r-markdown}

This is an R Markdown document. Markdown is a simple formatting syntax
for authoring HTML, PDF, and MS Word documents. For more details on
using R Markdown see \url{http://rmarkdown.rstudio.com}.

When you click the \textbf{Knit} button a document will be generated
that includes both content as well as the output of any embedded R code
chunks within the document. You can embed an R code chunk like this:

\begin{Shaded}
\begin{Highlighting}[]
\KeywordTok{summary}\NormalTok{(cars)}
\end{Highlighting}
\end{Shaded}

\begin{verbatim}
##      speed           dist       
##  Min.   : 4.0   Min.   :  2.00  
##  1st Qu.:12.0   1st Qu.: 26.00  
##  Median :15.0   Median : 36.00  
##  Mean   :15.4   Mean   : 42.98  
##  3rd Qu.:19.0   3rd Qu.: 56.00  
##  Max.   :25.0   Max.   :120.00
\end{verbatim}

\subsection{Including Plots}\label{including-plots}

You can also embed plots, for example:

\includegraphics{exemplo_files/figure-latex/pressure-1.pdf}

Note that the \texttt{echo\ =\ FALSE} parameter was added to the code
chunk to prevent printing of the R code that generated the plot.

\subsubsection{Nível de utilização geral de cada mídia
social}\label{nivel-de-utilizacao-geral-de-cada-midia-social}

A partir da formatação dos dados obtidos pelo questionário, conseguimos
identificar quais as mídias sociais mais utilizadas pelos entrevistados
e quais estão em maior desuso. Como está exposto no gráfico abaixo, as 3
mídias sociais mais utilizadas em ordem decrescente são: WhatsApp,
YouTube e Facebook.

Já as 3 mídias sociais menos utilizadas são: Tumblr, Snapchat e
Pinterest.

Interessante salientar que as mídias sociais informadas através da opção
``Outras'' foram: Telegram, Moodle, Steam, Skype, Messenger, GitHub,
Slack e StackOverFlow.

\includegraphics{exemplo_files/figure-latex/unnamed-chunk-3-1.pdf}

Desta forma, caso necessário realizar o compartilhamento de materiais,
por exemplo, entre docentes e discentes através de midias sociais
indica-se o compartilhamento de links em grupos de WhatsApp ou criação e
divulgação de vídeos pelos chamados canais da plataforma YouTube.

\subsubsection{Tempo de uso das mídias sociais por
dia}\label{tempo-de-uso-das-midias-sociais-por-dia}

\includegraphics{exemplo_files/figure-latex/unnamed-chunk-5-1.pdf}

\subsubsection{Opinião dos entrevistados sobre o uso de mídias sociais
por
professores}\label{opiniao-dos-entrevistados-sobre-o-uso-de-midias-sociais-por-professores}

Através da pergunta número 4 do formulário obteu-se a aceitação sobre do
uso de redes sociais pelos professores do Ensino Superior. Veja no
gráfico abaixo o compilado das respostas obtidas.

\includegraphics{exemplo_files/figure-latex/unnamed-chunk-9-1.pdf}

Baseado no resultado acima, percebemos o incentivo por parte dos
entrevistados (todos alunos do Ensino Superior) para que os professores
utilizem redes sociais. Entretanto, como a questão não menciona com qual
objetivo os professores utilizariam as redes sociais, ainda não é
possível concluir se os alunos apoiam o compartilhamento de informações
acadêmicas através das plataformas.

\subsubsection{Análise da possibilidade de integração entre docentes e
discentes através de redes
sociais}\label{analise-da-possibilidade-de-integracao-entre-docentes-e-discentes-atraves-de-redes-sociais}

Na pergunta de número 5 do questionário, os entrevistados foram
indagados se a melhor forma de aproximação entre alunos e professores é
através de redes sociais.

\includegraphics{exemplo_files/figure-latex/unnamed-chunk-11-1.pdf}

O resultado obtido acima é surpreendente. Uma vez que grande parte dos
alunos apoiam o uso de redes sociais pelos seus professores, o esperado
era que a maioria considerasse as mídias sociais como uma das melhores
formas de integração; o que não está visível no gráfico acima. Cerca de
22\% dos entrevistados preferem a aproximação de alunos aos seus
professores fora das redes sociais. Neste caso, seria interessante
aprofundarmos a questão para entender os motivos por esta escolha,
podendo envolver questões de privacidade ou insatisfação da mesclagem de
vida pessoal e vida acadêmica.

\subsubsection{Sobre a relação de melhores resultados de alunos com a
integração de mídias sociais às aulas e/ou
atividades}\label{sobre-a-relacao-de-melhores-resultados-de-alunos-com-a-integracao-de-midias-sociais-as-aulas-eou-atividades}

Na sexta pergunta, é questionada a possibilidade de alcance de melhores
resultados dos alunos, caso as redes sociais possuírem integração com as
aulas e atividades estudantis. Imagina-se o uso das plataformas para
divulgação de material de estudo, atividades e prazos para conclusão. \n
Veja em seguida os resultados.

\includegraphics{exemplo_files/figure-latex/unnamed-chunk-13-1.pdf}

\section{Percebemos que quase metade dos entrevistados julgam que os
alunos alcançariam melhores resultados com a integração de aulas com as
mídias
sociais.}\label{percebemos-que-quase-metade-dos-entrevistados-julgam-que-os-alunos-alcancariam-melhores-resultados-com-a-integracao-de-aulas-com-as-midias-sociais.}

\n
\#Portanto, mesmo que seja necessário encontrar outros meios de
aproximar alunos de seus professores, é recomendado o uso das
plataformas para proporcionar mais ganhos aos alunos.


\end{document}
